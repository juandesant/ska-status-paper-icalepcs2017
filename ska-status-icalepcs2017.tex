\documentclass[a4paper,
               % boxit,        % check whether paper is inside correct margins
               biblatex,       % biblatex is used
               keeplastbox,    % flushend option: not to un-indent last line in References
               % nospread,     % flushend option: do not fill with whitespace to balance columns
               % hyphens,      % allow \url to hyphenate at "-" (hyphens)
               % xetex,        % use XeLaTeX to process the file
               % luatex,       % use LuaLaTeX to process the file
               ]{jacow-2_1}    % uses jacow-2_1 to better support BibLaTeX
               
%
% CHANGE SEQUENCE OF GRAPHICS EXTENSION TO BE EMBEDDED
% ----------------------------------------------------
% test for XeTeX where the sequence is by default eps-> pdf, jpg, png, pdf, ...
%    and the JACoW template provides JACpic2v3.eps and JACpic2v3.jpg which
%    might generates errors, therefore PNG and JPG first
%
\makeatletter%
	\ifboolexpr{bool{xetex}}
	 {\renewcommand{\Gin@extensions}{.pdf,%
	                    .png,.jpg,.bmp,.pict,.tif,.psd,.mac,.sga,.tga,.gif,%
	                    .eps,.ps,%
	                    }}{}
\makeatother

% CHECK FOR XeTeX/LuaTeX BEFORE DEFINING AN INPUT ENCODING
% --------------------------------------------------------
%   utf8  is default for XeTeX/LuaTeX 
%   utf8  in LaTeX only realises a small portion of codes
%
\ifboolexpr{bool{xetex} or bool{luatex}} % test for XeTeX/LuaTeX
 {}                                      % input encoding is utf8 by default
 {\usepackage[utf8]{inputenc}}           % switch to utf8

\usepackage[USenglish]{babel}			 

\usepackage[final]{pdfpages}
\usepackage{multirow}
\usepackage{ragged2e}
 
\addbibresource{ska-status-icalepcs2017.bib}

%
% command for typesetting a \section like word
%
\newcommand\SEC[1]{\textbf{\uppercase{#1}}}

%%
%%   Lengths for the spaces in the title
%%   \setlength\titleblockstartskip{..}  %before title, default 3pt
%%   \setlength\titleblockmiddleskip{..} %between title + author, default 1em
%%   \setlength\titleblockendskip{..}    %afterauthor, default 1em

%\copyrightspace %default 1cm. arbitrary size with e.g. \copyrightspace[2cm]

% testing to fill the copyright space
%\usepackage{eso-pic}
%\AddToShipoutPictureFG*{\AtTextLowerLeft{\textcolor{red}{COPYRIGHTSPACE}}}


\begin{document}

\title{Status of the Square Kilometre Array}

\author{
	J. Santander-Vela\thanks{j.santander-vela@skatelescope.org},
	M. Bartolini, M. Deegan, L. Pivetta, N. Rees,\\
	SKA Organisation, Macclesfield, United Kingdom
}
	
\maketitle

%2345678901234567890123456789012345678901234567890123456789012345678901-
\begin{abstract}
	The Square Kilometre Array (SKA) is a international project to build a number of multi-purpose radio telescopes, operating as a single observatory, that will play a major role in answering key questions in modern astrophysics and cosmology. It will be one of a small number of cornerstone observatories around the world that will provide astrophysicists and cosmologists with a transformational view of the Universe. Two major goals of the SKA is to study the history and role of neutral Hydrogen in the Universe from the dark ages to the present-day, and to employ pulsars as probes of fundamental physics. Since 2008, the global radio astronomy community has been engaged in the development of the SKA and is now nearing the end of the \emph{Pre-Construction} phase. This talk provides an overview of the current status of the SKA and the plans for construction, focusing on the computing and software aspects of the project.
\end{abstract}


\section{Introduction} % (fold)
\label{sec:introduction}
The Square Kilometre Array (SKA) is an international project that has the aim of building multi-purposes radio telescopes, with an equivalent collecting area of at least one square kilometre, and thus unprecedented sensitivity, so that key questions in modern astrophysics and cosmology can be answered. The science cases that the SKA telescopes cover the Epoch of Reionisation, Cradle of Life, XXX, XXX. 

In 2015, \citetitle{2015aska.confE.....}~\cite{2015aska.confE.....} was published with more than 130 sicentific use cases that will be possible thanks to the SKA telescopes.

The SKA project is currently in what is known as SKA Phase 1, or SKA1, in which two telescopes approximately with 10\% of the target collecting area are being built, namely SKA1-Mid, and SKA1-Low, in order to prove the feasibility of the techniques and derisk the construction of the next phase of the project, SKA Phase 2, or SKA2.

This talk focuses on the progress and status of the SKA1 telescopes. It starts by describing the SKA Organisation itself (Sec.~\ref{sec:ska_organisation}), the current SKA timeline (Sec.~\ref{sec:ska1_timeline}), and the overall project organisation (Sec.~\ref{}). 

\subsection{SKA Organisation} % (fold)
\label{sub:ska_organisation}
The organisation overseing the SKA1 project is the SKA Organisation (SKAO), currently a limited liability non-for-profit company registered in England and Wales.

The SKAO is in the process of the becoming an Inter-Govermental treaty Organisation (IGO), not unlike the European Southern Observatory (ESO) or the European Council for Nuclear Research (CERN). The timeline for that process will be detailed in Sec.~\ref{sec:ska1_timeline}.

Currently\footnote{https://skatelescope.org/participating-countries/} there are ten countries as Full Members of the SKAO (in alfabetical order): Australia, Canada, China, India, Italy, New Zealand, South Africa, Sweden, The Netherlands, and United Kingdom. Other countries are involved in the design of the SKA1 telescopes, and it is estimated that \~20 countries and more than 100 organisations are contributed to that effort.

SKAO's headquartes are located within the boundaries of the Jodrell Bank Observatory, in the middle of the Cheshire plain, under direct view of the 70m Lovell Telescope, as shown in  Fig.~\ref{fig:SKA-HQ-at-Jodrell-Bank}.

\begin{figure}[!htb]
  \centering
    \includegraphics[width=\columnwidth]{SKA-HQ-aerial-panorama_web.jpg}
  \caption{The SKA Organisation Headquarters, at Jodrell Bank Observatory near Manchester, UK. The Lovell Telescope is to the left, Mark II to the right.}
  \label{fig:SKA-HQ-at-Jodrell-Bank}
\end{figure}

As part of the UK commitment as host country for the SKAO HQ, and the IGO, an expansion to the HQ is being constructed with the intention of becoming a nexus for radio astronomy. An artist rendition of the building can be found in Fig.~\ref{fig:SKA-HQ-render}, while the current status of the work, as of September 2017, can be seen in Fig.~\ref{fig:SKA-HQ2-aerial}.

\begin{figure}[!htb]
  \centering
    \includegraphics[width=\columnwidth]{SKA-HQ-render.jpg}
  \caption{Artist's view of the future SKAO HQ2, with the Council chamber in the foreground, and the Lovell telescope current Jodrell Bank Observatory building in the background.}
  \label{fig:SKA-HQ-render}
\end{figure}

\begin{figure}[!htb]
  \centering
    \includegraphics[width=\columnwidth]{SKA-HQ2-aerial.jpg}
  \caption{Aerial view of the current status of the SKAO HQ2 building, after the steel structure has been erected, and concrete slabs installed. The current SKAO HQ is in the foreground, Council chamber can be seen raising to the right.}
  \label{fig:SKA-HQ2-aerial}
\end{figure}


% subsection ska_organisation (end)

% section introduction (end)

\section{SKA1 Timeline} % (fold)
\label{sec:ska1_timeline}
The timeline 
% section ska1_timeline (end)

\section{Baseline Design} % (fold)
\label{sec:baseline_design}

\cite{SKA-TEL-SKO-0000002_v3} defines the baseline 
design capabilities of the SKA.


% section baseline_design (end)

\section{Conclusion} % (fold)
\label{sec:conclusion}
TBD. Any conclusions should be in a separate section directly preceding
the \SEC{Acknowledgement}, \SEC{Appendix}, or \SEC{References} sections, in that
order.

% section conclusion (end)

\section{Acknowledgement} % (fold)
\label{sec:acknowledgement}
TBD. Any acknowledgement should be in a separate section directly preceding
the \SEC{References} or \SEC{Appendix} section.

% section acknowledgement (end)

\section{Appendix} % (fold)
\label{sec:appendix}
TBD. Any appendix should be in a separate section directly preceding
the \SEC{References} section. If there is no \SEC{References} section,
this should be the last section of the paper.

% section appendix (end)


%\section*{References} % (fold)
\label{sec:references}
\printbibliography

% section references (end)




\end{document}
